\section{Struktur des Hauptdokumentes}

\begin{itemize}
 \item Wahl Art des Dokumentes mit \lstinline$\documentclass{}$, z.B.
       \lstinline$scrbook$, \lstinline$scrartcl$, \lstinline$beamer$, ...
 \item Einbinden verschiedener Pakete mit \lstinline$\usepackage[]{}$
 \item Inhalt umrahmt von \lstinline$\begin{document} ... \end{document}$
\end{itemize}

\begin{lstlisting}
\documentclass{scrartcl}
\usepackage[utf8]{inputenc}                 % Zeichencodierung
\usepackage{tabularx}                       % Tabellen
\usepackage{booktabs}                       % eleganter formatierte Tabellen
\usepackage{amsmath,amsfonts,amssymb}       % Matheumgebung und Symbole
\usepackage{graphicx}                       % Grafiken
\usepackage{units}                          % Setzen von Werten mit Einheiten
\usepackage{ngerman}                        % Verwende deutschen Zeilenumbruch

\begin{document}
 Inhalt
\end{document}
\end{lstlisting}

Der Inhalt kann, muss aber nicht im gleichen Dokument zu finden sein. Ab einer
Gesamtlänge des kompilierten Inhaltes von einer Seite, empfiehlt es sich, den Inhalt
in andere Dateien auszulagern und diese mit \lstinline$\input{}$ oder
\lstinline$\include{}$ einzubinden:

\begin{lstlisting}
%allerlei Pakete

\begin{document}

 \section{Struktur des Hauptdokumentes}

\begin{itemize}
 \item Wahl Art des Dokumentes mit \lstinline$\documentclass{}$, z.B.
       \lstinline$scrbook$, \lstinline$scrartcl$, \lstinline$beamer$, ...
 \item Einbinden verschiedener Pakete mit \lstinline$\usepackage[]{}$
 \item Inhalt umrahmt von \lstinline$\begin{document} ... \end{document}$
\end{itemize}

\begin{lstlisting}
\documentclass{scrartcl}
\usepackage[utf8]{inputenc}                 % Zeichencodierung
\usepackage{tabularx}                       % Tabellen
\usepackage{booktabs}                       % eleganter formatierte Tabellen
\usepackage{amsmath,amsfonts,amssymb}       % Matheumgebung und Symbole
\usepackage{graphicx}                       % Grafiken
\usepackage{units}                          % Setzen von Werten mit Einheiten
\usepackage{ngerman}                        % Verwende deutschen Zeilenumbruch

\begin{document}
 Inhalt
\end{document}
\end{lstlisting}

Der Inhalt kann, muss aber nicht im gleichen Dokument zu finden sein. Ab einer
Gesamtlänge des kompilierten Inhaltes von einer Seite, empfiehlt es sich, den Inhalt
in andere Dateien auszulagern und diese mit \lstinline$\input{}$ oder
\lstinline$\include{}$ einzubinden:

\begin{lstlisting}
%allerlei Pakete

\begin{document}

 \section{Struktur des Hauptdokumentes}

\begin{itemize}
 \item Wahl Art des Dokumentes mit \lstinline$\documentclass{}$, z.B.
       \lstinline$scrbook$, \lstinline$scrartcl$, \lstinline$beamer$, ...
 \item Einbinden verschiedener Pakete mit \lstinline$\usepackage[]{}$
 \item Inhalt umrahmt von \lstinline$\begin{document} ... \end{document}$
\end{itemize}

\begin{lstlisting}
\documentclass{scrartcl}
\usepackage[utf8]{inputenc}                 % Zeichencodierung
\usepackage{tabularx}                       % Tabellen
\usepackage{booktabs}                       % eleganter formatierte Tabellen
\usepackage{amsmath,amsfonts,amssymb}       % Matheumgebung und Symbole
\usepackage{graphicx}                       % Grafiken
\usepackage{units}                          % Setzen von Werten mit Einheiten
\usepackage{ngerman}                        % Verwende deutschen Zeilenumbruch

\begin{document}
 Inhalt
\end{document}
\end{lstlisting}

Der Inhalt kann, muss aber nicht im gleichen Dokument zu finden sein. Ab einer
Gesamtlänge des kompilierten Inhaltes von einer Seite, empfiehlt es sich, den Inhalt
in andere Dateien auszulagern und diese mit \lstinline$\input{}$ oder
\lstinline$\include{}$ einzubinden:

\begin{lstlisting}
%allerlei Pakete

\begin{document}

 \section{Struktur des Hauptdokumentes}

\begin{itemize}
 \item Wahl Art des Dokumentes mit \lstinline$\documentclass{}$, z.B.
       \lstinline$scrbook$, \lstinline$scrartcl$, \lstinline$beamer$, ...
 \item Einbinden verschiedener Pakete mit \lstinline$\usepackage[]{}$
 \item Inhalt umrahmt von \lstinline$\begin{document} ... \end{document}$
\end{itemize}

\begin{lstlisting}
\documentclass{scrartcl}
\usepackage[utf8]{inputenc}                 % Zeichencodierung
\usepackage{tabularx}                       % Tabellen
\usepackage{booktabs}                       % eleganter formatierte Tabellen
\usepackage{amsmath,amsfonts,amssymb}       % Matheumgebung und Symbole
\usepackage{graphicx}                       % Grafiken
\usepackage{units}                          % Setzen von Werten mit Einheiten
\usepackage{ngerman}                        % Verwende deutschen Zeilenumbruch

\begin{document}
 Inhalt
\end{document}
\end{lstlisting}

Der Inhalt kann, muss aber nicht im gleichen Dokument zu finden sein. Ab einer
Gesamtlänge des kompilierten Inhaltes von einer Seite, empfiehlt es sich, den Inhalt
in andere Dateien auszulagern und diese mit \lstinline$\input{}$ oder
\lstinline$\include{}$ einzubinden:

\begin{lstlisting}
%allerlei Pakete

\begin{document}

 \input{hauptdokument}                  % Datei hauptdokument.tex einfuegen
 \include{hauptdokument}                % auf neuer Seite einfuegen

\end{document}

\end{lstlisting}
                  % Datei hauptdokument.tex einfuegen
 \section{Struktur des Hauptdokumentes}

\begin{itemize}
 \item Wahl Art des Dokumentes mit \lstinline$\documentclass{}$, z.B.
       \lstinline$scrbook$, \lstinline$scrartcl$, \lstinline$beamer$, ...
 \item Einbinden verschiedener Pakete mit \lstinline$\usepackage[]{}$
 \item Inhalt umrahmt von \lstinline$\begin{document} ... \end{document}$
\end{itemize}

\begin{lstlisting}
\documentclass{scrartcl}
\usepackage[utf8]{inputenc}                 % Zeichencodierung
\usepackage{tabularx}                       % Tabellen
\usepackage{booktabs}                       % eleganter formatierte Tabellen
\usepackage{amsmath,amsfonts,amssymb}       % Matheumgebung und Symbole
\usepackage{graphicx}                       % Grafiken
\usepackage{units}                          % Setzen von Werten mit Einheiten
\usepackage{ngerman}                        % Verwende deutschen Zeilenumbruch

\begin{document}
 Inhalt
\end{document}
\end{lstlisting}

Der Inhalt kann, muss aber nicht im gleichen Dokument zu finden sein. Ab einer
Gesamtlänge des kompilierten Inhaltes von einer Seite, empfiehlt es sich, den Inhalt
in andere Dateien auszulagern und diese mit \lstinline$\input{}$ oder
\lstinline$\include{}$ einzubinden:

\begin{lstlisting}
%allerlei Pakete

\begin{document}

 \input{hauptdokument}                  % Datei hauptdokument.tex einfuegen
 \include{hauptdokument}                % auf neuer Seite einfuegen

\end{document}

\end{lstlisting}
                % auf neuer Seite einfuegen

\end{document}

\end{lstlisting}
                  % Datei hauptdokument.tex einfuegen
 \section{Struktur des Hauptdokumentes}

\begin{itemize}
 \item Wahl Art des Dokumentes mit \lstinline$\documentclass{}$, z.B.
       \lstinline$scrbook$, \lstinline$scrartcl$, \lstinline$beamer$, ...
 \item Einbinden verschiedener Pakete mit \lstinline$\usepackage[]{}$
 \item Inhalt umrahmt von \lstinline$\begin{document} ... \end{document}$
\end{itemize}

\begin{lstlisting}
\documentclass{scrartcl}
\usepackage[utf8]{inputenc}                 % Zeichencodierung
\usepackage{tabularx}                       % Tabellen
\usepackage{booktabs}                       % eleganter formatierte Tabellen
\usepackage{amsmath,amsfonts,amssymb}       % Matheumgebung und Symbole
\usepackage{graphicx}                       % Grafiken
\usepackage{units}                          % Setzen von Werten mit Einheiten
\usepackage{ngerman}                        % Verwende deutschen Zeilenumbruch

\begin{document}
 Inhalt
\end{document}
\end{lstlisting}

Der Inhalt kann, muss aber nicht im gleichen Dokument zu finden sein. Ab einer
Gesamtlänge des kompilierten Inhaltes von einer Seite, empfiehlt es sich, den Inhalt
in andere Dateien auszulagern und diese mit \lstinline$\input{}$ oder
\lstinline$\include{}$ einzubinden:

\begin{lstlisting}
%allerlei Pakete

\begin{document}

 \section{Struktur des Hauptdokumentes}

\begin{itemize}
 \item Wahl Art des Dokumentes mit \lstinline$\documentclass{}$, z.B.
       \lstinline$scrbook$, \lstinline$scrartcl$, \lstinline$beamer$, ...
 \item Einbinden verschiedener Pakete mit \lstinline$\usepackage[]{}$
 \item Inhalt umrahmt von \lstinline$\begin{document} ... \end{document}$
\end{itemize}

\begin{lstlisting}
\documentclass{scrartcl}
\usepackage[utf8]{inputenc}                 % Zeichencodierung
\usepackage{tabularx}                       % Tabellen
\usepackage{booktabs}                       % eleganter formatierte Tabellen
\usepackage{amsmath,amsfonts,amssymb}       % Matheumgebung und Symbole
\usepackage{graphicx}                       % Grafiken
\usepackage{units}                          % Setzen von Werten mit Einheiten
\usepackage{ngerman}                        % Verwende deutschen Zeilenumbruch

\begin{document}
 Inhalt
\end{document}
\end{lstlisting}

Der Inhalt kann, muss aber nicht im gleichen Dokument zu finden sein. Ab einer
Gesamtlänge des kompilierten Inhaltes von einer Seite, empfiehlt es sich, den Inhalt
in andere Dateien auszulagern und diese mit \lstinline$\input{}$ oder
\lstinline$\include{}$ einzubinden:

\begin{lstlisting}
%allerlei Pakete

\begin{document}

 \input{hauptdokument}                  % Datei hauptdokument.tex einfuegen
 \include{hauptdokument}                % auf neuer Seite einfuegen

\end{document}

\end{lstlisting}
                  % Datei hauptdokument.tex einfuegen
 \section{Struktur des Hauptdokumentes}

\begin{itemize}
 \item Wahl Art des Dokumentes mit \lstinline$\documentclass{}$, z.B.
       \lstinline$scrbook$, \lstinline$scrartcl$, \lstinline$beamer$, ...
 \item Einbinden verschiedener Pakete mit \lstinline$\usepackage[]{}$
 \item Inhalt umrahmt von \lstinline$\begin{document} ... \end{document}$
\end{itemize}

\begin{lstlisting}
\documentclass{scrartcl}
\usepackage[utf8]{inputenc}                 % Zeichencodierung
\usepackage{tabularx}                       % Tabellen
\usepackage{booktabs}                       % eleganter formatierte Tabellen
\usepackage{amsmath,amsfonts,amssymb}       % Matheumgebung und Symbole
\usepackage{graphicx}                       % Grafiken
\usepackage{units}                          % Setzen von Werten mit Einheiten
\usepackage{ngerman}                        % Verwende deutschen Zeilenumbruch

\begin{document}
 Inhalt
\end{document}
\end{lstlisting}

Der Inhalt kann, muss aber nicht im gleichen Dokument zu finden sein. Ab einer
Gesamtlänge des kompilierten Inhaltes von einer Seite, empfiehlt es sich, den Inhalt
in andere Dateien auszulagern und diese mit \lstinline$\input{}$ oder
\lstinline$\include{}$ einzubinden:

\begin{lstlisting}
%allerlei Pakete

\begin{document}

 \input{hauptdokument}                  % Datei hauptdokument.tex einfuegen
 \include{hauptdokument}                % auf neuer Seite einfuegen

\end{document}

\end{lstlisting}
                % auf neuer Seite einfuegen

\end{document}

\end{lstlisting}
                % auf neuer Seite einfuegen

\end{document}

\end{lstlisting}
                  % Datei hauptdokument.tex einfuegen
 \section{Struktur des Hauptdokumentes}

\begin{itemize}
 \item Wahl Art des Dokumentes mit \lstinline$\documentclass{}$, z.B.
       \lstinline$scrbook$, \lstinline$scrartcl$, \lstinline$beamer$, ...
 \item Einbinden verschiedener Pakete mit \lstinline$\usepackage[]{}$
 \item Inhalt umrahmt von \lstinline$\begin{document} ... \end{document}$
\end{itemize}

\begin{lstlisting}
\documentclass{scrartcl}
\usepackage[utf8]{inputenc}                 % Zeichencodierung
\usepackage{tabularx}                       % Tabellen
\usepackage{booktabs}                       % eleganter formatierte Tabellen
\usepackage{amsmath,amsfonts,amssymb}       % Matheumgebung und Symbole
\usepackage{graphicx}                       % Grafiken
\usepackage{units}                          % Setzen von Werten mit Einheiten
\usepackage{ngerman}                        % Verwende deutschen Zeilenumbruch

\begin{document}
 Inhalt
\end{document}
\end{lstlisting}

Der Inhalt kann, muss aber nicht im gleichen Dokument zu finden sein. Ab einer
Gesamtlänge des kompilierten Inhaltes von einer Seite, empfiehlt es sich, den Inhalt
in andere Dateien auszulagern und diese mit \lstinline$\input{}$ oder
\lstinline$\include{}$ einzubinden:

\begin{lstlisting}
%allerlei Pakete

\begin{document}

 \section{Struktur des Hauptdokumentes}

\begin{itemize}
 \item Wahl Art des Dokumentes mit \lstinline$\documentclass{}$, z.B.
       \lstinline$scrbook$, \lstinline$scrartcl$, \lstinline$beamer$, ...
 \item Einbinden verschiedener Pakete mit \lstinline$\usepackage[]{}$
 \item Inhalt umrahmt von \lstinline$\begin{document} ... \end{document}$
\end{itemize}

\begin{lstlisting}
\documentclass{scrartcl}
\usepackage[utf8]{inputenc}                 % Zeichencodierung
\usepackage{tabularx}                       % Tabellen
\usepackage{booktabs}                       % eleganter formatierte Tabellen
\usepackage{amsmath,amsfonts,amssymb}       % Matheumgebung und Symbole
\usepackage{graphicx}                       % Grafiken
\usepackage{units}                          % Setzen von Werten mit Einheiten
\usepackage{ngerman}                        % Verwende deutschen Zeilenumbruch

\begin{document}
 Inhalt
\end{document}
\end{lstlisting}

Der Inhalt kann, muss aber nicht im gleichen Dokument zu finden sein. Ab einer
Gesamtlänge des kompilierten Inhaltes von einer Seite, empfiehlt es sich, den Inhalt
in andere Dateien auszulagern und diese mit \lstinline$\input{}$ oder
\lstinline$\include{}$ einzubinden:

\begin{lstlisting}
%allerlei Pakete

\begin{document}

 \section{Struktur des Hauptdokumentes}

\begin{itemize}
 \item Wahl Art des Dokumentes mit \lstinline$\documentclass{}$, z.B.
       \lstinline$scrbook$, \lstinline$scrartcl$, \lstinline$beamer$, ...
 \item Einbinden verschiedener Pakete mit \lstinline$\usepackage[]{}$
 \item Inhalt umrahmt von \lstinline$\begin{document} ... \end{document}$
\end{itemize}

\begin{lstlisting}
\documentclass{scrartcl}
\usepackage[utf8]{inputenc}                 % Zeichencodierung
\usepackage{tabularx}                       % Tabellen
\usepackage{booktabs}                       % eleganter formatierte Tabellen
\usepackage{amsmath,amsfonts,amssymb}       % Matheumgebung und Symbole
\usepackage{graphicx}                       % Grafiken
\usepackage{units}                          % Setzen von Werten mit Einheiten
\usepackage{ngerman}                        % Verwende deutschen Zeilenumbruch

\begin{document}
 Inhalt
\end{document}
\end{lstlisting}

Der Inhalt kann, muss aber nicht im gleichen Dokument zu finden sein. Ab einer
Gesamtlänge des kompilierten Inhaltes von einer Seite, empfiehlt es sich, den Inhalt
in andere Dateien auszulagern und diese mit \lstinline$\input{}$ oder
\lstinline$\include{}$ einzubinden:

\begin{lstlisting}
%allerlei Pakete

\begin{document}

 \input{hauptdokument}                  % Datei hauptdokument.tex einfuegen
 \include{hauptdokument}                % auf neuer Seite einfuegen

\end{document}

\end{lstlisting}
                  % Datei hauptdokument.tex einfuegen
 \section{Struktur des Hauptdokumentes}

\begin{itemize}
 \item Wahl Art des Dokumentes mit \lstinline$\documentclass{}$, z.B.
       \lstinline$scrbook$, \lstinline$scrartcl$, \lstinline$beamer$, ...
 \item Einbinden verschiedener Pakete mit \lstinline$\usepackage[]{}$
 \item Inhalt umrahmt von \lstinline$\begin{document} ... \end{document}$
\end{itemize}

\begin{lstlisting}
\documentclass{scrartcl}
\usepackage[utf8]{inputenc}                 % Zeichencodierung
\usepackage{tabularx}                       % Tabellen
\usepackage{booktabs}                       % eleganter formatierte Tabellen
\usepackage{amsmath,amsfonts,amssymb}       % Matheumgebung und Symbole
\usepackage{graphicx}                       % Grafiken
\usepackage{units}                          % Setzen von Werten mit Einheiten
\usepackage{ngerman}                        % Verwende deutschen Zeilenumbruch

\begin{document}
 Inhalt
\end{document}
\end{lstlisting}

Der Inhalt kann, muss aber nicht im gleichen Dokument zu finden sein. Ab einer
Gesamtlänge des kompilierten Inhaltes von einer Seite, empfiehlt es sich, den Inhalt
in andere Dateien auszulagern und diese mit \lstinline$\input{}$ oder
\lstinline$\include{}$ einzubinden:

\begin{lstlisting}
%allerlei Pakete

\begin{document}

 \input{hauptdokument}                  % Datei hauptdokument.tex einfuegen
 \include{hauptdokument}                % auf neuer Seite einfuegen

\end{document}

\end{lstlisting}
                % auf neuer Seite einfuegen

\end{document}

\end{lstlisting}
                  % Datei hauptdokument.tex einfuegen
 \section{Struktur des Hauptdokumentes}

\begin{itemize}
 \item Wahl Art des Dokumentes mit \lstinline$\documentclass{}$, z.B.
       \lstinline$scrbook$, \lstinline$scrartcl$, \lstinline$beamer$, ...
 \item Einbinden verschiedener Pakete mit \lstinline$\usepackage[]{}$
 \item Inhalt umrahmt von \lstinline$\begin{document} ... \end{document}$
\end{itemize}

\begin{lstlisting}
\documentclass{scrartcl}
\usepackage[utf8]{inputenc}                 % Zeichencodierung
\usepackage{tabularx}                       % Tabellen
\usepackage{booktabs}                       % eleganter formatierte Tabellen
\usepackage{amsmath,amsfonts,amssymb}       % Matheumgebung und Symbole
\usepackage{graphicx}                       % Grafiken
\usepackage{units}                          % Setzen von Werten mit Einheiten
\usepackage{ngerman}                        % Verwende deutschen Zeilenumbruch

\begin{document}
 Inhalt
\end{document}
\end{lstlisting}

Der Inhalt kann, muss aber nicht im gleichen Dokument zu finden sein. Ab einer
Gesamtlänge des kompilierten Inhaltes von einer Seite, empfiehlt es sich, den Inhalt
in andere Dateien auszulagern und diese mit \lstinline$\input{}$ oder
\lstinline$\include{}$ einzubinden:

\begin{lstlisting}
%allerlei Pakete

\begin{document}

 \section{Struktur des Hauptdokumentes}

\begin{itemize}
 \item Wahl Art des Dokumentes mit \lstinline$\documentclass{}$, z.B.
       \lstinline$scrbook$, \lstinline$scrartcl$, \lstinline$beamer$, ...
 \item Einbinden verschiedener Pakete mit \lstinline$\usepackage[]{}$
 \item Inhalt umrahmt von \lstinline$\begin{document} ... \end{document}$
\end{itemize}

\begin{lstlisting}
\documentclass{scrartcl}
\usepackage[utf8]{inputenc}                 % Zeichencodierung
\usepackage{tabularx}                       % Tabellen
\usepackage{booktabs}                       % eleganter formatierte Tabellen
\usepackage{amsmath,amsfonts,amssymb}       % Matheumgebung und Symbole
\usepackage{graphicx}                       % Grafiken
\usepackage{units}                          % Setzen von Werten mit Einheiten
\usepackage{ngerman}                        % Verwende deutschen Zeilenumbruch

\begin{document}
 Inhalt
\end{document}
\end{lstlisting}

Der Inhalt kann, muss aber nicht im gleichen Dokument zu finden sein. Ab einer
Gesamtlänge des kompilierten Inhaltes von einer Seite, empfiehlt es sich, den Inhalt
in andere Dateien auszulagern und diese mit \lstinline$\input{}$ oder
\lstinline$\include{}$ einzubinden:

\begin{lstlisting}
%allerlei Pakete

\begin{document}

 \input{hauptdokument}                  % Datei hauptdokument.tex einfuegen
 \include{hauptdokument}                % auf neuer Seite einfuegen

\end{document}

\end{lstlisting}
                  % Datei hauptdokument.tex einfuegen
 \section{Struktur des Hauptdokumentes}

\begin{itemize}
 \item Wahl Art des Dokumentes mit \lstinline$\documentclass{}$, z.B.
       \lstinline$scrbook$, \lstinline$scrartcl$, \lstinline$beamer$, ...
 \item Einbinden verschiedener Pakete mit \lstinline$\usepackage[]{}$
 \item Inhalt umrahmt von \lstinline$\begin{document} ... \end{document}$
\end{itemize}

\begin{lstlisting}
\documentclass{scrartcl}
\usepackage[utf8]{inputenc}                 % Zeichencodierung
\usepackage{tabularx}                       % Tabellen
\usepackage{booktabs}                       % eleganter formatierte Tabellen
\usepackage{amsmath,amsfonts,amssymb}       % Matheumgebung und Symbole
\usepackage{graphicx}                       % Grafiken
\usepackage{units}                          % Setzen von Werten mit Einheiten
\usepackage{ngerman}                        % Verwende deutschen Zeilenumbruch

\begin{document}
 Inhalt
\end{document}
\end{lstlisting}

Der Inhalt kann, muss aber nicht im gleichen Dokument zu finden sein. Ab einer
Gesamtlänge des kompilierten Inhaltes von einer Seite, empfiehlt es sich, den Inhalt
in andere Dateien auszulagern und diese mit \lstinline$\input{}$ oder
\lstinline$\include{}$ einzubinden:

\begin{lstlisting}
%allerlei Pakete

\begin{document}

 \input{hauptdokument}                  % Datei hauptdokument.tex einfuegen
 \include{hauptdokument}                % auf neuer Seite einfuegen

\end{document}

\end{lstlisting}
                % auf neuer Seite einfuegen

\end{document}

\end{lstlisting}
                % auf neuer Seite einfuegen

\end{document}

\end{lstlisting}
                % auf neuer Seite einfuegen

\end{document}

\end{lstlisting}
