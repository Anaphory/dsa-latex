\section{Formelsatz}

\subsection*{In einer Formelumgebung}
abgesetzte, (nummerierte), referenzierbare Formel

\begin{lstlisting}
\begin{equation}
 \label{Satz_des_Pythagoras}      % optional
 a^2 + b^2 = c^2
\end{equation}
\end{lstlisting}

\begin{equation}
 \label{Satz_des_Pythagoras}      % optional
 a^2 + b^2 = c^2
\end{equation}

\begin{lstlisting}
\begin{equation*}
 c \le a + b
\end{equation*}
\end{lstlisting}

\begin{equation*}
 c \le a + b
\end{equation*}

\subsection*{Im Fließtext}
Einrahmung der Formel durch \$Formel\$

\begin{lstlisting}
Aus dem Satz des Pythagoras (siehe Gleichung \ref{Satz_des_Pythagoras}) laesst
sich die Laenge der Hypotenuse zu $c = \sqrt{a^2 + b^2}$ bestimmen.
\end{lstlisting}

Aus dem Satz des Pythagoras (siehe Gleichung \ref{Satz_des_Pythagoras}) lässt
sich die Länge der Hypotenuse zu $c = \sqrt{a^2 + b^2}$ bestimmen.


\subsection{Chemische Formeln im Fließtext}
\begin{itemize}
 \item Elementsymbole werden in normaler Schrift geschrieben: H, C, O, N, Xe, ...
 \item Anzahlen und Ladungen werden in der Matheumgebung notiert
\end{itemize}

\begin{lstlisting}
Bei der Autoprotolyse des Wassers entstehen Hydroniumionen H$_{3}$O$^{+}$ und
Hydroxidionen OH$^{-}$.
\end{lstlisting}

Bei der Autoprotolyse des Wassers entstehen Hydroniumionen H$_{3}$O$^{+}$ und
Hydroxidionen OH$^{-}$.

