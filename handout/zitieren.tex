\section{Zitieren mit bibtex}

\begin{itemize}
 \item Einträge in Name.bib Datei verwalten
 \item Literaturbibliothek in Hauptdatei einbauen
 \item mit \lstinline$\cite{}$ zitieren
\end{itemize}

Hauptdatei:
\begin{lstlisting}
%Pakete, Inhalte, usw.
\section{Zitieren mit bibtex}

\begin{itemize}
 \item Einträge in Name.bib Datei verwalten
 \item Literaturbibliothek in Hauptdatei einbauen
 \item mit \lstinline$\cite{}$ zitieren
\end{itemize}

Hauptdatei:
\begin{lstlisting}
%Pakete, Inhalte, usw.
\section{Zitieren mit bibtex}

\begin{itemize}
 \item Einträge in Name.bib Datei verwalten
 \item Literaturbibliothek in Hauptdatei einbauen
 \item mit \lstinline$\cite{}$ zitieren
\end{itemize}

Hauptdatei:
\begin{lstlisting}
%Pakete, Inhalte, usw.
\section{Zitieren mit bibtex}

\begin{itemize}
 \item Einträge in Name.bib Datei verwalten
 \item Literaturbibliothek in Hauptdatei einbauen
 \item mit \lstinline$\cite{}$ zitieren
\end{itemize}

Hauptdatei:
\begin{lstlisting}
%Pakete, Inhalte, usw.
\input{zitieren}

\bibliographystyle{unsrtdin}
\bibliography{lit2}

\end{document}
\end{lstlisting}

Bibliothekendatei Name.bib (hier lit2.bib)
\begin{lstlisting}
@ARTICLE{Fasshauer13,
  author = {Fasshauer, E. and Pernpointner, M. and Gokhberg, K.},
  title = {Interatomic Decay of Inner-Valence Ionized States in ArXe Clusters:
        Relativistic Approach},
  journal = {J. Chem. Phys.},
  year = {2013},
  volume = {138},
  pages = {014305},
}

@BOOK{SakuraiModern94,
  title = {Modern Quantum Mechanics},
  publisher = {Addison-Wesley},
  year = {1994},
  editor = {Tuan, S. F.},
  author = {Sakurai, J. J.},
  edition = {Rev.},
}

@MISC{Nobelpreis,
  author = {Various Artist},
  title = {Development of the metathesis method in organic synthesis},
        url = {http://www.nobelprize.org/nobel\_prizes/chemistry/laureates/2005/advanced-chemistryprize2005.pdf},
        howpublished = {http://www.nobelprize.org/nobel\_prizes/ chemistry/laureates/2005/advanced-chemistryprize2005.pdf},
}
\end{lstlisting}

\begin{lstlisting}
Diese Eintraege sind im Fliesstext selber nicht zu sehen. Wenn man jedoch das
Buch \cite{SakuraiModern94}, den Artikel \cite{Fasshauer13} 
oder den Onlineartikel \cite{Nobelpreis} zitiert,
werden die Nummern automatisch eingefuegt und ein Literaturverzeichnis am Ende
des Dokumentes erstellt.
\end{lstlisting}

Diese Einträge sind im Fließtext selber nicht zu sehen. Wenn man jedoch das
Buch \cite{SakuraiModern94}, den Artikel \cite{Fasshauer13} oder den
Onlineartikel \cite{Nobelpreis} zitiert,
werden die Nummern automatisch eingefügt und ein Literaturverzeichnis am Ende
des Dokumentes erstellt.


\bibliographystyle{unsrtdin}
\bibliography{lit2}

\end{document}
\end{lstlisting}

Bibliothekendatei Name.bib (hier lit2.bib)
\begin{lstlisting}
@ARTICLE{Fasshauer13,
  author = {Fasshauer, E. and Pernpointner, M. and Gokhberg, K.},
  title = {Interatomic Decay of Inner-Valence Ionized States in ArXe Clusters:
        Relativistic Approach},
  journal = {J. Chem. Phys.},
  year = {2013},
  volume = {138},
  pages = {014305},
}

@BOOK{SakuraiModern94,
  title = {Modern Quantum Mechanics},
  publisher = {Addison-Wesley},
  year = {1994},
  editor = {Tuan, S. F.},
  author = {Sakurai, J. J.},
  edition = {Rev.},
}

@MISC{Nobelpreis,
  author = {Various Artist},
  title = {Development of the metathesis method in organic synthesis},
        url = {http://www.nobelprize.org/nobel\_prizes/chemistry/laureates/2005/advanced-chemistryprize2005.pdf},
        howpublished = {http://www.nobelprize.org/nobel\_prizes/ chemistry/laureates/2005/advanced-chemistryprize2005.pdf},
}
\end{lstlisting}

\begin{lstlisting}
Diese Eintraege sind im Fliesstext selber nicht zu sehen. Wenn man jedoch das
Buch \cite{SakuraiModern94}, den Artikel \cite{Fasshauer13} 
oder den Onlineartikel \cite{Nobelpreis} zitiert,
werden die Nummern automatisch eingefuegt und ein Literaturverzeichnis am Ende
des Dokumentes erstellt.
\end{lstlisting}

Diese Einträge sind im Fließtext selber nicht zu sehen. Wenn man jedoch das
Buch \cite{SakuraiModern94}, den Artikel \cite{Fasshauer13} oder den
Onlineartikel \cite{Nobelpreis} zitiert,
werden die Nummern automatisch eingefügt und ein Literaturverzeichnis am Ende
des Dokumentes erstellt.


\bibliographystyle{unsrtdin}
\bibliography{lit2}

\end{document}
\end{lstlisting}

Bibliothekendatei Name.bib (hier lit2.bib)
\begin{lstlisting}
@ARTICLE{Fasshauer13,
  author = {Fasshauer, E. and Pernpointner, M. and Gokhberg, K.},
  title = {Interatomic Decay of Inner-Valence Ionized States in ArXe Clusters:
        Relativistic Approach},
  journal = {J. Chem. Phys.},
  year = {2013},
  volume = {138},
  pages = {014305},
}

@BOOK{SakuraiModern94,
  title = {Modern Quantum Mechanics},
  publisher = {Addison-Wesley},
  year = {1994},
  editor = {Tuan, S. F.},
  author = {Sakurai, J. J.},
  edition = {Rev.},
}

@MISC{Nobelpreis,
  author = {Various Artist},
  title = {Development of the metathesis method in organic synthesis},
        url = {http://www.nobelprize.org/nobel\_prizes/chemistry/laureates/2005/advanced-chemistryprize2005.pdf},
        howpublished = {http://www.nobelprize.org/nobel\_prizes/ chemistry/laureates/2005/advanced-chemistryprize2005.pdf},
}
\end{lstlisting}

\begin{lstlisting}
Diese Eintraege sind im Fliesstext selber nicht zu sehen. Wenn man jedoch das
Buch \cite{SakuraiModern94}, den Artikel \cite{Fasshauer13} 
oder den Onlineartikel \cite{Nobelpreis} zitiert,
werden die Nummern automatisch eingefuegt und ein Literaturverzeichnis am Ende
des Dokumentes erstellt.
\end{lstlisting}

Diese Einträge sind im Fließtext selber nicht zu sehen. Wenn man jedoch das
Buch \cite{SakuraiModern94}, den Artikel \cite{Fasshauer13} oder den
Onlineartikel \cite{Nobelpreis} zitiert,
werden die Nummern automatisch eingefügt und ein Literaturverzeichnis am Ende
des Dokumentes erstellt.


\bibliographystyle{unsrtdin}
\bibliography{lit2}

\end{document}
\end{lstlisting}

Bibliothekendatei Name.bib (hier lit2.bib)
\begin{lstlisting}
@ARTICLE{Fasshauer13,
  author = {Fasshauer, E. and Pernpointner, M. and Gokhberg, K.},
  title = {Interatomic Decay of Inner-Valence Ionized States in ArXe Clusters:
        Relativistic Approach},
  journal = {J. Chem. Phys.},
  year = {2013},
  volume = {138},
  pages = {014305},
}

@BOOK{SakuraiModern94,
  title = {Modern Quantum Mechanics},
  publisher = {Addison-Wesley},
  year = {1994},
  editor = {Tuan, S. F.},
  author = {Sakurai, J. J.},
  edition = {Rev.},
}

@MISC{Nobelpreis,
  author = {Various Artist},
  title = {Development of the metathesis method in organic synthesis},
        url = {http://www.nobelprize.org/nobel\_prizes/chemistry/laureates/2005/advanced-chemistryprize2005.pdf},
        howpublished = {http://www.nobelprize.org/nobel\_prizes/ chemistry/laureates/2005/advanced-chemistryprize2005.pdf},
}
\end{lstlisting}

\begin{lstlisting}
Diese Eintraege sind im Fliesstext selber nicht zu sehen. Wenn man jedoch das
Buch \cite{SakuraiModern94}, den Artikel \cite{Fasshauer13} 
oder den Onlineartikel \cite{Nobelpreis} zitiert,
werden die Nummern automatisch eingefuegt und ein Literaturverzeichnis am Ende
des Dokumentes erstellt.
\end{lstlisting}

Diese Einträge sind im Fließtext selber nicht zu sehen. Wenn man jedoch das
Buch \cite{SakuraiModern94}, den Artikel \cite{Fasshauer13} oder den
Onlineartikel \cite{Nobelpreis} zitiert,
werden die Nummern automatisch eingefügt und ein Literaturverzeichnis am Ende
des Dokumentes erstellt.
